% Activate the following line by filling in the right side. If for example the name of the root file is Main.tex, write
% "...root = Main.tex" if the chapter file is in the same directory, and "...root = ../Main.tex" if the chapter is in a subdirectory.
 
%!TEX root =  dissertation.tex

\chapter[Conclusion]{Conclusion}


Computational fluid dynamics simulation platforms like HemeLB are inherently complex. Domain experts, whose knowledge in the subject is invaluable, often face difficulties in using the software package. This complexity creates a barrier for them to run a study or even an exploratory use of the software. In addition to that, this barrier is also detrimental towards the replicable and reproducible aspect of the discipline. In this project, I tried to solve this problem by providing an alternative interface for HemeLB. 

With HemeWeb, I provide another example to the likes of Nekkloud and Galaxy projects on how a web application can be used as an alternative interface to complex application. In addition, this web application coupled with Docker enables it to provide a reproducible research platform for complex applications which is valuable to science in general. I believe using web application and cloud platform is the trend forward for complex applications like HemeLB. For example, at 10th August 2016, The VMTK Lab team announced that they will release a "Computational Fluid Dynamic on the Cloud" tools for their cardiovascular modeling application, VMTKLab, at 20th September 2016. The tools they add to the application allow users to run on-demand simulation using cloud infrastructure\citep{VMTKL63:online}. This is in line with what this project is about and shows that there is a company solving similar demands of on-demand computational platform beyond the academia sector.


In this project, I designed and implemented HemeWeb, a web-based interface to interact with the HemeLB simulation workflow. It is developed to answer the problems above. Originally, users are required to use command line interface, but HemeWeb allows them to use a web browser to configure and run HemeLB simulation. 

In developing HemeWeb, I managed to produce three software outputs. These are HemeLB core Docker container, HemeWeb deployment scripts, and HemeWeb web application. All of these components are developed openly and are publicly available on the public repository of this project\footnote{\url{https://github.com/SeiryuZ/HemeWeb}}. In addition,HemeLB core Docker container is also published on the Docker hub. All of these enables an interested party to get their hands on the project to experiment or work with it.

The deployment scripts can configure HemeLB ready infrastructure on three cloud vendors. They are Digital Ocean, Amazon Web Service, and Google Cloud Platform. With it, users can configure cloud platform's instances to handle multi-host HemeLB simulation via command line interface. However, currently, the HemeWeb web application can only be deployed with Amazon Web Service. This is caused by an abstraction layer that currently only supports Amazon Web Service for a certain feature like simulation files persistence.

The HemeWeb web application itself allows domain experts to use the familiar interface of clicking on a web form to configure and run HemeLB simulation. Users can provide pre-processed input files or even unprocessed input files to the web app, and it will allow users to configure the job and run it within few clicks. In addition to that, HemeWeb also allows users to use past simulation information, both owned locally or remotely, to create a new simulation. This capability allows users to either replicate or reproduce a simulation.

From the survey, we found  that the respondents are considerably satisfied with the system. HemeWeb allows the respondents to run a simulation and reproduce a simulation with the exception of two respondents which skip the scenarios. In doing both of these tasks, the respondents are considerably satisfied with the ease of doing it, the time it takes, and the support HemeWeb give them. In addition, some of the respondents find using command line interface is a barrier, so this provides a nice alternatives for them to run a HemeLB simulation. In Chapter 7, I will outline potential improvements for HemeWeb that includes feedback from this survey.

Performance wise, HemeWeb perform respectably, especially in lower core count, when compared against dedicated infrastructure for HPC application like INDY2 and ARCHER supercomputer. AWS-EC2 even perform slightly better than INDY2 machine at one instance during our evaluation. However, AWS-EC2 do not scale as well as the dedicated HPC infrastructure.

Next, we calculated the notional cost of running a simulation job on AWS-EC2 and ARCHER with two different pricing. Running HemeLB simulation on AWS-EC2 is 230-290\% more expensive when compared to ARCHER without partner pricing, and 682-846\% more expensive with partner pricing. The cost of simulation grow with regards to the increase of compute nodes used for the simulation. 

Based on the performance analysis done on Chapter 5, we identified the ideal use-case that benefits from a cloud-enabled solution. This ideal type of job is a one-off job that is usually an exploratory type of simulation. Being a one-off job, means that users do not need constant compute units. This use-case benefits from the flexibility that cloud platform like Amazon provide compared to going through a grant application to gain access to institutional or regional HPC infrastructure.

With regards to costs, while it is currently more expensive, companies have known to cut prices of their services in the pasts\citep{AWSPr74:online, Annou90:online, Googl18:online}, making it potentially more cheaper to use their service in the near future. This makes HemeWeb an attractive alternative interface to run HemeLB simulation for scientists, especially those who does not have access to dedicated HPC infrastructures. 








