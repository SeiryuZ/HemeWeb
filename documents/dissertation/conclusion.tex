% Activate the following line by filling in the right side. If for example the name of the root file is Main.tex, write
% "...root = Main.tex" if the chapter file is in the same directory, and "...root = ../Main.tex" if the chapter is in a subdirectory.
 
%!TEX root =  dissertation.tex

\chapter[Conclusion]{Conclusion}


Computational bioinformatic applications like HemeLB is inherently complex. Domain experts, whose knowledge in the subject is invaluable, often face difficulties in using the software package. This complexity creates a barrier for them to run a study or even an exploratory use of the software. In addition to that, this barrier is also detrimental towards the replicable and reproducible aspect of the discipline. In this project, I alleviate this pain by providing an alternative interface for HemeLB. With it, I provide another example to the likes of Nekkloud and Galaxy projects on how a web application can be used as an alternative interface to complex application. In addition, this web application coupled with Docker enables it to provide a reproducible research platform for complex applications which is valuable to science in general. I believe using web application and cloud platform is the trend forward for complex applications like HemeLB. For example, at 10th August 2016, The VMTK Lab team announced that they will release a "Computational Fluid Dynamic on the Cloud" tools for their cardiovascular modeling application, VMTKLab, at 20th September 2016. The tools they add to the application allow users to run on-demand simulation using cloud infrastructure\citep{VMTKL63:online}. This is in line with what this project is about and shows that there are people trying to solve similar problem.


In this project, I designed and implemented HemeWeb, a web-based interface to interact with HemeLB simulation workflow. It is developed to answer the problems above. Originally, users are required to use command line interface, but HemeWeb allows them to use a web browser to configure and run HemeLB simulation. 

In developing HemeWeb, I managed to produce 3 outputs. These are HemeLB core Docker container, HemeWeb deployment scripts, and HemeWeb web application. In the spirit of being open, all of these components are open-sourced and publicly available on the public repository of this project\footnote{\url{https://github.com/SeiryuZ/HemeWeb}}. HemeLB core Docker container is also published on the Docker hub. All of these enables an interested party to get their hands on the project to experiment or work with it.

The deployment scripts can configure HemeLB ready infrastructure on three cloud vendors. They are Digital Ocean, Amazon Web Service, and Google Cloud Platform. With it, users can configure cloud platform's instances to handle multi-host HemeLB simulation via command line interface. However, currently, the HemeWeb web application can only be deployed with Amazon Web Service. This is caused by an abstraction layer that currently only supports Amazon Web Service for a certain feature like simulation files persistence.

The HemeWeb web application itself allows domain experts to use the familiar interface of clicking on a web form to configure and run HemeLB simulation. Users can provide pre-processed input files or even unprocessed input files to the web app, and it will allow users to configure the job and run it within few clicks. In addition to that, HemeWeb also allows users to use past simulation information, both owned locally or remotely, to create a new simulation. This capability allows users to either replicate or reproduce a simulation.

From the survey, we found  that the respondents are considerably satisfied with the system. HemeWeb allows the respondents to run a simulation and reproduce a simulation with the exception of one respondent. In doing both of these tasks, the respondents are considerably satisfied with the ease of doing it, the time it takes, and the support HemeWeb give them. Overall, HemeWeb is not perfect. There are things that could be better, especially the interface. However, considering that the user satisfactions are subjective, the respondent's sentiment for HemeWeb are generally good. In addition, some of the respondents find using command line interface is a barrier, so this provides a nice alternatives for them to run a HemeLB simulation.

Performance wise, HemeWeb perform worse up to 11.27 slower at its worst when compared against dedicated infrastructure for HPC application like INDY2 and ARCHER supercomputer. In addition, costs more than renting those HPC infrastructure resources and this will be worse when considering the job will finish much slower. However, this performance penalty should not be the only consideration in decision to use HemeWeb. HemeWeb offers flexibility in the way domain experts use computing infrastructures. One should only have a credit card and they can start running experiments. In addition to that, domain experts or institutions does not have to worry about maintaining said infrastructure if they are building it themselves. Another reason is that while currently, cloud platforms cost more, companies have cut prices of their services in the pasts\citep{AWSPr74:online, Annou90:online, Googl18:online}, making it potentially more enticing and easy to use their service compared to dedicated infrastructure in the near future.



