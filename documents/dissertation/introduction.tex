% Activate the following line by filling in the right side. If for example the name of the root file is Main.tex, write
% "...root = Main.tex" if the chapter file is in the same directory, and "...root = ../Main.tex" if the chapter is in a subdirectory.
 
% !TEX root =  dissertation.tex

\chapter[Introduction]{Introduction}



%Software is increasingly complex. Our everyday software is packed with features that make its usage difficult. To people without familiarity with the product, complex usage can be a barrier to use the software even when it can help them tremendously. 
%
%This also ties into the complexity of a research that use this software. Open science dictates that research should be reproducible or replicable for it to better validate the research. However, recent findings have shown that not many research in psychology are replicable.  


\section{Motivation}
To study how blood flow in a given vessel, Mazzeo and Coveney\cite{mazzeo2008hemelb} developed a fluid dynamic simulation software named HemeLB. Currently, it is actively developed and used by researchers to help their study. For example, Itani et al.\cite{itani2015automated} used HemeLB for automated ensemble simulation of blood flow for a range of exercise intensities,  Bernabeu et al.\cite{bernabeu2015characterization} used it for detecting the difference of retinal hemodynamics with regards to diabetic retinopathy, and recently Franco et al.\cite{franco2015dynamic,franco2016non} used it to understand branching pattern of blood vessel networks.

As I have written in the proposal for this dissertation \citep{Steven:2016aa}, HemeLB works by calculating fluid flow in parallel by using Lattice-Boltzmann method \citep{mazzeo2008hemelb}. This calculation allows HemeLB to simulate blood flow within a given blood vessel structure. Unfortunately, the calculation part is only a small part of the workflow to run the simulation. There are multiple pre-processing and post-processing steps needed to run the simulation from start to the end. These includes of preparing the input so HemeLB can work on it, and also processing the output so it is ready to view.

These long pipelines of steps needed to run the simulation, coupled with the complexity of the configuration of HemeLB created a high barrier of entry for scientists and doctors to use it. Furthermore, as observed previously \citep{Steven:2016aa}, an interesting simulation will require parallel computing resources like ARCHER supercomputer which might be difficult to get access to by interested parties. While smaller simulation instances can run on a typical laptop, most of the problems will require more powerful machines.These facts might prevent usage of the software by interested parties. More importantly, it shows there are still improvements that can be done to lower the barrier of entry for users to use HemeLB. This is important for HemeLB, especially when it is envisioned to be part of future medical decision \citep{1_green_2014}.

Another aspect that HemeLB workflow can be improved is with regards to its reproducibility aspect. Researches that used HemeLB embrace reproducibility as one of its concern. As observed before \citep{Steven:2016aa}, There are steps that are in place to make sure HemeLB and its simulation result are reproducible. First, the entire code base is publicly available on Github. Second, in running a simulation with HemeLB, version of the software is automatically recorded. Lastly, in addition to the version used, input files and configurations are also recorded automatically. These facts can be seen from the publications mentioned above \citep{bernabeu2015characterization,itani2015automated,franco2015dynamic,franco2016non} that include all these information. This information allow researchers interested to replicate the simulation to do it manually. Automation of these steps could further improve HemeLB's reproducibility and allow peers to replicate, duplicate, and audit its simulation results quickly and easily. This automation will be important, in addition to being usable, for HemeLB to become an integral part of the medical decision in the future.

\section{Objectives}

Based on the needs to improve the usability. reproducibility, and auditability aspect of HemeLB project, I will develop a prototype web interface for HemeLB. This prototype web interface will lower barrier of entry in using HemeLB software compared to the current approach of using command line interface. In addition to that, using web interface will also allow features to be added to the simulation workflow which might not be essential to the HemeLB core itself. For example, automating packaging, sharing, and reproducing simulation result. These features are not essential for the HemeLB core, but definitely, help the overall workflow of blood flow simulation.

Using the dynamic capabilities of cloud computing vendor, the web backend should be able to dynamically scale without many efforts. On top of that, these infrastructures are available to everyone with a cost, allowing its user to access it without having to get access to supercomputers. Its user should be able to run a blood flow simulation without having to deal with the complexity of running each step of the workflow manually.

In addition to the web interface, I will also develop deployment script so that peer could deploy its own instance of the web interface. This will ease up deployment process for individuals or organizations intending to use HemeLB for its own purposes. This script will be developed as part of the project.



\section{Outline}
I provide a brief introduction to the topic of this dissertation in this chapter. The rest of the chapters will be organized as follow:
\begin{itemize}
    \item \textbf{Chapter 2, Background}. I will provide background information that is necessary for readers to understand the concepts, technology, and implementation that are done in this dissertation. HemeLB, containerization technology, cloud computing, High-Performance computing infrastructure, and other topics will be discussed in details in this chapter.
    \item \textbf{Chapter 3, Design and Implementation}. I will discuss the bulk of the work in this chapter. Implementation details and design of the proposed solutions will be provided and discussed in details.
    \item \textbf{Chapter 4, Evaluation}. In this chapter, I will discuss how the success of this project will be measured. I detailed how I conduct an interview and performance benchmark to measure how the proposed solution perform.
    \item \textbf{Chapter 5, Result and Analysis}. I will discuss my findings from the evaluation of the project. 
    \item \textbf{Chapter 6, Conclusion}. I will conclude and summarize the whole project.
    \item \textbf{Chapter 7, Future work}. Some recommendations on the future of the project in the context of HemeLB simulation workflow.
\end{itemize}
