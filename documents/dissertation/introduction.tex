% Activate the following line by filling in the right side. If for example the name of the root file is Main.tex, write
% "...root = Main.tex" if the chapter file is in the same directory, and "...root = ../Main.tex" if the chapter is in a subdirectory.
 
% !TEX root =  dissertation.tex

\chapter[Introduction]{Introduction}



%Software is increasingly complex. Our everyday software is packed with features that make its usage difficult. To people without familiarity with the product, complex usage can be a barrier to use the software even when it can help them tremendously. 
%
%This also ties into the complexity of a research that use this software. Open science dictates that research should be reproducible or replicable for it to better validate the research. However, recent findings have shown that not many research in psychology are replicable.  


\section{Motivation}
To study how blood flows in a given vessel, Mazzeo and Coveney \cite{mazzeo2008hemelb} developed a fluid dynamic simulation software named HemeLB. Currently, it is actively developed and used by researchers to help their study. For example, Itani et al. \cite{itani2015automated} used HemeLB for automated ensemble simulation of blood flow for a range of exercise intensities,  Bernabeu et al. \cite{bernabeu2015characterization} used it for detecting the difference in retinal hemodynamics with regards to diabetic retinopathy, and recently Franco et al. \cite{franco2015dynamic,franco2016non} used it to understand the branching pattern of blood vessel networks.

As I have written in the proposal for this dissertation \citep{Steven:2016aa}, HemeLB works by calculating fluid flow in parallel with using the Lattice-Boltzmann method \citep{mazzeo2008hemelb}. This calculation allows HemeLB to simulate blood flow within a given blood vessel structure. Inevitably, the calculation is only a small part of the workflow to run the simulation. There are multiple pre-processing and post-processing steps needed to run the simulation from start to end. These include preparing the input so HemeLB can work on it, and also processing the output so it is ready to view or further analysis.

These long pipelines of steps needed to run the simulation, coupled with the complexity of the configuration of HemeLB created a high barrier of entry for scientists and doctors to use it. Furthermore, as observed previously \citep{Steven:2016aa}, an interesting simulation will require parallel computing resources like the ARCHER supercomputer which might be difficult to get access to by interested parties. While smaller simulation instances can run on a typical laptop, most of the problems will require more powerful machines. These facts might prevent usage of the software by domain experts. More importantly, it shows there are still improvements that can be made to lower the barrier of entry for users to use HemeLB. This is important for HemeLB, especially when it is envisioned to be part of future medical decisions \citep{1_green_2014}.

Another aspect of the HemeLB workflow that can be improved is with regards to its reproducibility and replicability aspect. Chris Drummond \citep{drummond2009replicability} make a distinction between these two terms by arguing that reproducibility require changes to the original experiment, while replicability avoid changes to it. Research projects that used HemeLB embraces these concepts as their priority. There are steps that are in place to make sure HemeLB and its simulation result are reproducible. First, the entire code base is publicly available on Github. Second, in running a simulation with HemeLB, a version of the software is automatically recorded. Lastly, in addition to the version used, input files and configurations are also recorded automatically. These facts can be seen from the publications mentioned above \citep{bernabeu2015characterization,itani2015automated,franco2015dynamic,franco2016non} that include all this information. This information allows researchers interested to replicate the simulation or tweak the simulation parameters to run their own simulation. Automation of these steps could further improve HemeLB's reproducibility and allow peers to replicate, duplicate, and audit its simulation results quickly and easily. 


\section{Objectives}

Based on the need to improve the usability and reproducibility aspects of the HemeLB project, I will develop a prototype web interface for HemeLB. This prototype web interface will lower the barrier of entry in using HemeLB software compared to the current approach of using the Command Line Interface(CLI). Currently, the approach consists of manual installation of HemeLB onto the HPC infrastructure, copying data into it, running simulation, and extracting outputs. These steps require users to be proficient with the CLI and are hard for people without the appropriate skills.  By using a web interface, it will allow domain experts to do all these steps in a more-familiar setting of web browser. In addition, it will also allow features to be added to the simulation workflow which might not be essential to the HemeLB core itself. For example, automating packaging, sharing, and reproducing the simulation result. These features are not essential but useful for helping the overall workflow of blood flow simulation.

Using the dynamic capabilities of the cloud computing vendor, the web backend should be able to dynamically scale without much effort. On top of that, these infrastructures are available to everyone for a cost, can be started on demand, and user will only pay what were used. These benefits allow user to rent computing resources to run HemeLB simulation without having to get access to dedicated HPC infrastructure. The web backend will allow users to run a blood flow simulation without having to deal with the complexity of running each step of the workflow manually.

In addition to the web interface, I will also develop a deployment script so that peers can deploy their own instance of the web interface. This will ease up the deployment process for individuals or organizations intending to use HemeLB for their own purposes. This script will be developed as part of the project.



\section{Outline}
I provide a brief introduction to the topic of this dissertation in this chapter. The rest of the chapters will be organized as follows:
\begin{itemize}
    \item \textbf{Chapter 2, Background}. I will provide background information that is necessary for readers to understand the concepts, technology, and implementation in this dissertation. HemeLB, containerization technology, cloud computing, High-Performance computing infrastructure, and other topics will be discussed in detail in this chapter.
    \item \textbf{Chapter 3, Design and Implementation}. I will discuss the bulk of the work in this chapter. The implementation details and design of the proposed solutions will be provided and discussed in detail.
    \item \textbf{Chapter 4, Evaluation}. In this chapter, I will discuss how the success of this project will be measured. I detail how I conduct a usability survey and use performance benchmarks to measure how the proposed solution performs.
    \item \textbf{Chapter 5, Result and Analysis}. I will discuss my findings from the evaluation of the project. 
    \item \textbf{Chapter 6, Conclusion}. I will conclude and summarize the whole project.
    \item \textbf{Chapter 7, Future Work}. Some recommendations on the future of the project in the context of the HemeLB simulation workflow.
\end{itemize}
