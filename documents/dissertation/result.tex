% Activate the following line by filling in the right side. If for example the name of the root file is Main.tex, write
% "...root = Main.tex" if the chapter file is in the same directory, and "...root = ../Main.tex" if the chapter is in a subdirectory.
 
%!TEX root =  dissertation.tex

\chapter[Analysis]{Analysis}

\section{Usability result and analysis}

Usability evaluation is done using online survey hosted by Google Form\footnote{\url{https://goo.gl/forms/toYsRwnGIGumMBUD2}}. Respondents are given 2 tasks to complete, which are to run a simulation and reproduce past simulation. I analyze respondent's answers with regards to HemeWeb's usability.


\subsection{Demography}


\vspace{1cm}

\noindent%
\begin{minipage}{\linewidth}% to keep image and caption on one page
\makebox[\linewidth]{
  \includegraphics[keepaspectratio=true,scale=0.8]{../resources/evaluation/usability/career.png}
 }
\captionof{figure}{Career stage distribution} \label{fig:survey-career}%      only if needed  
\end{minipage}

\vspace{1cm}

\vspace{1cm}

\noindent%
\begin{minipage}{\linewidth}% to keep image and caption on one page
\makebox[\linewidth]{
  \includegraphics[keepaspectratio=true,scale=0.8]{../resources/evaluation/usability/discipline.png}
 }
\captionof{figure}{Career discipline} \label{fig:survey-discipline}%      only if needed  
\end{minipage}

\vspace{1cm}


The survey was filled by 16 respondents over the period of 10 days (3rd August 2016 - 10th August 2016). However, because 1 of the respondent are unable to run the tasks, we remove the reponse because it will not add any meaningful information. In the end we have 15 valid respondents. We can classify the respondents with their career stage and discipline. From here, we can classify the respondents from informatics / computer science related discipline or not. From figure \ref{fig:survey-career} and \ref{fig:survey-discipline}, around 62.5\% of the respondents are either informatics or computational scientists, while 37.3\% can be considered domain experts (Biologist, Clinician, and Biophysicist)

\vspace{1cm}

\noindent%
\begin{minipage}{\linewidth}% to keep image and caption on one page
\makebox[\linewidth]{
  \includegraphics[keepaspectratio=true,scale=0.8]{../resources/evaluation/usability/source_code.png}
 }
\captionof{figure}{Familiarity with installing software from source code} \label{fig:survey-source}%      only if needed  
\end{minipage}

\vspace{1cm}

\noindent%
\begin{minipage}{\linewidth}% to keep image and caption on one page
\makebox[\linewidth]{
  \includegraphics[keepaspectratio=true,scale=0.8]{../resources/evaluation/usability/browser.png}
 }
\captionof{figure}{Familiarity with web browser} \label{fig:survey-browser}%      only if needed  
\end{minipage}

\vspace{1cm}

\noindent%
\begin{minipage}{\linewidth}% to keep image and caption on one page
\makebox[\linewidth]{
  \includegraphics[keepaspectratio=true,scale=0.8]{../resources/evaluation/usability/hemelb.png}
 }
\captionof{figure}{Familiarity with CFD tools like HemeLB} \label{fig:survey-hemelb}%      only if needed  
\end{minipage}

\vspace{1cm}


We can also further classify the respondents based on the familiarity with browsers, computational fluid dynamic tools, and installing software from source code. All of this are shown in Figure \ref{fig:survey-source}, \ref{fig:survey-browser}, and \ref{fig:survey-hemelb}.




\subsection{Scenario 1: Run a simulation}

To measure the usability of HemeWeb for users to run a simulation, respondents were asked to run a  scenario. Instructions to run a HemeLB simulation were provided and they were asked to follow it to produce a simulation. Respondents then were asked to state their agreement with three statements which are provided. These three questions measure how usable HemeWeb is to them. 


\vspace{1cm}

\noindent%
\begin{minipage}{\linewidth}% to keep image and caption on one page
\makebox[\linewidth]{
  \includegraphics[keepaspectratio=true,scale=0.9]{../resources/evaluation/usability/scenario1_usability.png}
 }
\captionof{figure}{Scenario 1 usability} \label{fig:survey-s1-usability}%      only if needed  
\end{minipage}

On Figure \ref{fig:survey-s1-usability}, the responses is shown. Generally, respondents tend to agree that HemeWeb are usable to run a simulation from the three statements. However, one respondent fill out Not applicable towards the statement that says they are satisfied with support information HemeWeb give them. This could means that the respondent doesn't feel the scenario give enough information for them to agree or disagree with the statements.

Based on the responses, we can also deduce the system's usability in running a simulation by calculating the After Scenario Questionnare score, or ASQ score. We assign an integer value, 1 for Strongly disagree, 2 for disagree, 3 for neutral, 4 for Agree, and 5 for strongly disagree. Next, we take the mean of the values for each questions and ignore the answer with Not Applicable to determine a single ASQ score for that respondent. With this schema we can determine that for the respondents tend to agree that they are satisfied with the usability of the system with the overall ASQ score of 4.36

Our hypothesis is that users will generally find using web interface is usable and the data seemed to support that. It is much easier for user to complete a task when using point and click interface rather than requiring them to recall commands to do the tasks, especially when they are not familiar with the tools.


Next, respondents are given a high overview of replicating the tasks done in HemeWeb but using command line interface. Respondents are not asked to run the scenario in their command line interface because we cannot make sure the necessary tools are installed on respondent's computer.

\vspace{1cm}

\noindent%
\begin{minipage}{\linewidth}% to keep image and caption on one page
\makebox[\linewidth]{
  \includegraphics[keepaspectratio=true,scale=0.9]{../resources/evaluation/usability/scenario1_cli.png}
 }
\captionof{figure}{Scenario 1 command line preference} \label{fig:survey-s1-cli}%      only if needed  
\end{minipage}

\vspace{1cm}

\noindent%
\begin{minipage}{\linewidth}% to keep image and caption on one page
\makebox[\linewidth]{
  \includegraphics[keepaspectratio=true,scale=0.9]{../resources/evaluation/usability/scenario1_cli_barrier.png}
 }
\captionof{figure}{Scenario 1 barrier in using command line} \label{fig:survey-s1-cli-barrier}%      only if needed  
\end{minipage}

\vspace{1cm}

Figure \ref{fig:survey-s1-cli} and \ref{fig:survey-s1-cli-barrier} shows the general sentiment of users in running the simulation but in the command line. From the data, respondents are generally open to the likelihood of them running the simulation using command line, if we quantify the response, it got the score of 3.46, somewhere between neutral and likely. However, when we realize that the the 62.5\% of the respondent has background in informatics / computational scientists who deal a lot with command line, this become clearer.

Our hypothesis is that domain experts would most likely find CLI a barrier to run a simulation and ess likely to run simulation with it.



\subsection{Scenario 2: Reproduce past simulation}

\vspace{1cm}

\noindent%
\begin{minipage}{\linewidth}% to keep image and caption on one page
\makebox[\linewidth]{
  \includegraphics[keepaspectratio=true,scale=0.9]{../resources/evaluation/usability/scenario2_usability.png}
 }
\captionof{figure}{Scenario 2 usability} \label{fig:survey-s2-usability}%      only if needed  
\end{minipage}

\vspace{1cm}

\noindent%
\begin{minipage}{\linewidth}% to keep image and caption on one page
\makebox[\linewidth]{
  \includegraphics[keepaspectratio=true,scale=0.9]{../resources/evaluation/usability/scenario2_cli.png}
 }
\captionof{figure}{Scenario 2 command line preference} \label{fig:survey-s2-cli}%      only if needed  
\end{minipage}

\vspace{1cm}

\noindent%
\begin{minipage}{\linewidth}% to keep image and caption on one page
\makebox[\linewidth]{
  \includegraphics[keepaspectratio=true,scale=0.9]{../resources/evaluation/usability/scenario2_cli_barrier.png}
 }
\captionof{figure}{Scenario 2 barrier in using command line} \label{fig:survey-s2-cli-barrier}%      only if needed  
\end{minipage}

\vspace{1cm}



\subsection{Overall usability}


%%%%%%%%%%%%%%%%%%%%%%%%%%%%%%%%%%%
%  SYSTEM USEFULNESS 
%%%%%%%%%%%%%%%%%%%%%%%%%%%%%%%%%%%
\begin{center}
\captionof{table}{System usefulness}\label{table:overall-usability}
\scalebox{0.75}{

\begin{tabular}{|l|l|l|l|l|l|l|}
\hline
                                                                                                          & \begin{tabular}[c]{@{}l@{}}Strongly \\ disagree\end{tabular} & Disagree & Neutral & Agree & \begin{tabular}[c]{@{}l@{}}Strongly \\ agree\end{tabular} & \begin{tabular}[c]{@{}l@{}}Not \\ applicable\end{tabular} \\ \hline
\begin{tabular}[c]{@{}l@{}}Overall, I am satisfied with\\  how easy it is to use this system\end{tabular} & 0                                                            & 0        & 0       & 8     & 7                                                         & 0                                                         \\ \hline
It was simple to use this system                                                                          & 0                                                            & 0        & 0       & 2     & 13                                                        & 0                                                         \\ \hline
I can effectively complete my work using this system                                                      & 1                                                            & 1        & 2       & 3     & 4                                                         & 4                                                         \\ \hline
I am able to complete my work quickly using this system                                                   & 0                                                            & 1        & 3       & 3     & 6                                                         & 2                                                         \\ \hline
I am able to efficiently complete my work using this system                                               & 0                                                            & 2        & 2       & 4     & 5                                                         & 2                                                         \\ \hline
I feel comfortable using this system                                                                      & 1                                                            & 1        & 0       & 3     & 10                                                        & 0                                                         \\ \hline
It was easy to learn to use this system                                                                   & 0                                                            & 0        & 2       & 0     & 13                                                        & 0                                                         \\ \hline
I believe I became productive quickly using this system                                                   & 0                                                            & 1        & 2       & 3     & 5                                                         & 4                                                         \\ \hline
\end{tabular}
}
\end{center}
\vspace{1cm}


%%%%%%%%%%%%%%%%%%%%%%%%%%%%%%%%%%%
%  INFO QUALITY
%%%%%%%%%%%%%%%%%%%%%%%%%%%%%%%%%%%

\begin{center}
\captionof{table}{Information quality}\label{table:overall-info-quality}
\scalebox{0.75}{
\begin{tabular}{|l|l|l|l|l|l|l|}
\hline
                                                                                                                                                                               & \begin{tabular}[c]{@{}l@{}}Strongly\\ disagree\end{tabular} & Disagree & Neutral & Agree & \begin{tabular}[c]{@{}l@{}}Strongly\\ agree\end{tabular} & \begin{tabular}[c]{@{}l@{}}Not\\ applicable\end{tabular} \\ \hline
\begin{tabular}[c]{@{}l@{}}The system gives error messages that\\  clearly tell me how to fix problems\end{tabular}                                                            & 0                                                           & 2        & 1       & 4     & 2                                                        & 6                                                        \\ \hline
\begin{tabular}[c]{@{}l@{}}Whenever I make a mistake using the system, \\ I recover easily and quickly\end{tabular}                                                            & 0                                                           & 0        & 5       & 1     & 3                                                        & 6                                                        \\ \hline
\begin{tabular}[c]{@{}l@{}}The information (such as online help, \\ on-screen messages, and other documentation)\\  provided with this system is clear\end{tabular}            & 0                                                           & 0        & 3       & 4     & 5                                                        & 3                                                        \\ \hline
It is easy to find the information I needed                                                                                                                                    & 0                                                           & 1        & 2       & 3     & 6                                                        & 3                                                        \\ \hline
\begin{tabular}[c]{@{}l@{}}The information (such as online help,\\  on-screen messages, and other documentation)\\  provided for the system is easy to understand\end{tabular} & 0                                                           & 1        & 2       & 5     & 6                                                        & 1                                                        \\ \hline
\begin{tabular}[c]{@{}l@{}}The information is effective in helping me\\  complete the tasks and scenarios\end{tabular}                                                         & 0                                                           & 1        & 4       & 2     & 7                                                        & 1                                                        \\ \hline
\begin{tabular}[c]{@{}l@{}}The organization of information on the \\ system screens is clear\end{tabular}                                                                      & 0                                                           & 0        & 3       & 8     & 4                                                        & 0                                                        \\ \hline
\end{tabular}

}
\end{center}
\vspace{1cm}



%%%%%%%%%%%%%%%%%%%%%%%%%%%%%%%%%%%
%  INTERFACE QUALITY
%%%%%%%%%%%%%%%%%%%%%%%%%%%%%%%%%%%
\begin{center}
\captionof{table}{Interface quality}\label{table:overall-interface-quality}
\scalebox{0.75}{
\begin{tabular}{|l|l|l|l|l|l|l|}
\hline
                                                                                                                  & \begin{tabular}[c]{@{}l@{}}Strongly\\ disagree\end{tabular} & Disagree & Neutral & Agree & \begin{tabular}[c]{@{}l@{}}Strongly\\ agree\end{tabular} & \begin{tabular}[c]{@{}l@{}}Not\\ applicable\end{tabular} \\ \hline
The interface of this system is pleasant                                                                          & 1                                                           & 1        & 1       & 7     & 5                                                        & 0                                                        \\ \hline
I like using the interface of this system                                                                         & 0                                                           & 2        & 1       & 8     & 4                                                        & 0                                                        \\ \hline
\begin{tabular}[c]{@{}l@{}}This system has all the functions and\\  capabilities I expect it to have\end{tabular} & 1                                                           & 2        & 1       & 4     & 4                                                        & 3                                                        \\ \hline
Overall, I am satisfied with this system                                                                          & 0                                                           & 0        & 2       & 7     & 6                                                        & 0                                                        \\ \hline
\end{tabular}
}
\end{center}
\vspace{1cm}

\section{Performance result and analysis}

In this section, I will discuss the performance benchmark of HemeLB simulation done in ARCHER supercomputer, Indy2 HPC cluster, and AWS EC2 where HemeWeb is deployed to. HemeLB produced a report files that can be used to benchmark the performance of the simulation execution. I made sure that on each infrastructure, the input file we used is the same. I took the simulation total time result from that files and compare the value between infrastructure.


\begin{center}

\captionof{table}{Performance comparison HemeLB on Indy2 vs AWS EC2}\label{table:perf}

\begin{tabular}{|c|c|c|}
\hline
\multicolumn{1}{|l|}{}            & \multicolumn{2}{c|}{Performance in seconds}               \\ \hline
\multicolumn{1}{|c|}{\# of Cores} & \multicolumn{1}{c|}{Indy2} & \multicolumn{1}{c|}{AWS EC2} \\ \hline
36                                & 24.7                       & 36.3                         \\ \hline
72                                & 12.7                       & 29.9                         \\ \hline
144                               & 7.08                       & 36.5                         \\ \hline
288                               & 3.44                       & 32.4                         \\ \hline
576                               & 1.81                       & 20.4                         \\ \hline
1152                              & 1.58                       & N/A                             \\ \hline
\end{tabular}

\end{center}



\vspace{1cm}

\begin{center}
\captionof{table}{HemeLB performance on ARCHER supercomputer}\label{table:perf-archer}
\begin{tabular}{|c|c|c|}
\hline
\multicolumn{1}{|l|}{}            & \multicolumn{1}{c|}{Performance in seconds}               \\ \hline
\multicolumn{1}{|l|}{\# of Cores} & \multicolumn{1}{c|}{ARCHER}  \\ \hline
24                                & 24.7                                           \\ \hline
48                                & 12.7                                         \\ \hline
96                               & 7.08                                       \\ \hline
192                               & 3.44                                           \\ \hline
384                               & 1.81                                               \\ \hline
768                              & 1.58                                              \\ \hline
1536                              & 1.58                                              \\ \hline
\end{tabular}
\end{center}


\vspace{1cm}

\noindent%
\begin{minipage}{\linewidth}% to keep image and caption on one page
\makebox[\linewidth]{
  \includegraphics[keepaspectratio=true,scale=0.75]{../resources/evaluation/performance/overview.png}
 }
\captionof{figure}{HemeLB performance comparison} \label{fig:hemelb-perf-overview}%      only if needed  
\end{minipage}

\vspace{1cm}


In general, the performance of HemeLB with cloud infrastructure is worse when compared to ARCHER and Indy2. On each simulation instances with a different number of cores, we see slower simulation execution result when compared to the likes of the Indy2 machine and ARCHER supercomputer as observed on table \ref{table:perf}, \ref{table:perf-archer} and figure \ref{fig:hemelb-perf-overview}.

ARCHER supercomputer has slower simulation time compared to Indy2, because of the difference of processors used in the compute node. ARCHER used a three-year-old 2.7 GHz, 12-core E5-2697 v2 Ivy Bridge processor. Compared to Indy2 which boasts the newer Broadwell-based Intel(R) Xeon(R) CPU E5-2695 v4 @ 2.10GHz. The number of thread inside those processors also differ, ARCHER has 24 while Indy2 has 36. This makes the performance difference. On the other hand for this evaluation, we use Amazon's c4.8xlarge EC2 instance which has Haswell-based E5-2666 v3 processor that has 36 virtual CPU cores. All these difference contributes toward the speed difference oh the simulation results.

The scaling of the performance is where HemeWeb took a dive. It is apparent that with increased compute node, the network activity between the nodes become a bottleneck in the HemeWeb's case. The performance dip when we started the simulation using 4 compute nodes which have 144 cores that it is slower than just using 1 compute node. However, the performance improves as we add more compute nodes. This is not seen in the case of Indy2, where it scales very well and reach a diminishing return on the performance after using more than 1,000 cores.




\section {Limitation of the evaluations}

\subsection{Usability evaluation}

When observing the evaluation, one might argue that 5 point scale used in the questionnaire is not enough as pointed out by Kraig Finstad \cite{finstad2010response}. In his research, he argued that 5 point scale for a questionnaire allows more room for respondents to interpolate their response. He compared the same version of usability study but with five-point and seven-point scale and found out that 3\% of the respondents answer to five point scale actually are interpolation, while the seven point scale have no interpolation. He concluded that administering usability study using seven point scale will achieve greater accuracy. However, I decided to use the five point scale of the questionnaire because of the limitation of google survey platform. 

\noindent%
\begin{minipage}{\linewidth}% to keep image and caption on one page
\makebox[\linewidth]{
  \includegraphics[keepaspectratio=true,scale=0.35]{../resources/images/google-limitation.png}
 }
\captionof{figure}{Google form hiding some options on 7 point scale} \label{fig:google-limit}%      only if needed  
\end{minipage}


\vspace{1cm}
\noindent%
\begin{minipage}{\linewidth}% to keep image and caption on one page
\makebox[\linewidth]{
  \includegraphics[keepaspectratio=true,scale=0.35]{../resources/images/google-limitation2.png}
 }
\captionof{figure}{Google form render 5 point scale rating better} \label{fig:google-limit2}%      only if needed  
\end{minipage}

\vspace{1cm}


Figure \ref{fig:google-limit} show the limitation of the platform. The response options are not rendered in complete, as in some are hidden in a scrolling element of the page. This leads to respondents have to scroll left and right to see the options and answers. I find this is a detriment to the respondents to do more work in answering the question so I decided to use 5 point scale to make sure everything is rendered better like shown in Figure \ref{fig:google-limit2}.


Another limitation of the study is the way the study compare the experience of doing the scenarios in the questionnaire between using web browser and using command line.  When measuring the usability of HemeWeb, respondents are asked to do the task in the web browser, compared to the task in command line where it is just described in an overview. This decision is made because we cannot make sure the respondents have the necessary tools to run or reproduce a simulation in their computers.  In addition, in order to run the scenario in command line interface, respondents have to install tools on their computer which are timely and prone to errors. I believe this will affect their sentiment in answering the survey questionnaire and decided against it.


\subsection{Performance evaluation}

On running the HemeLB simulation on the three systems compared, we cannot fully isolate the infrastructure from various system load the infrastructure is handling. The performance might be affected by other jobs running in the system, which is apparent in the case of ARCHER supercomputer where many jobs are executing at the same moment. The same case with Indy2. The performance benchmark might be not as accurate as a completely isolated environment, however it paints a pretty good picture of performance you get when using cloud vendors compared to the dedicated infrastructure. 
 
