% Activate the following line by filling in the right side. If for example the name of the root file is Main.tex, write
% "...root = Main.tex" if the chapter file is in the same directory, and "...root = ../Main.tex" if the chapter is in a subdirectory.
 
%!TEX root =  dissertation.tex

\chapter[Analysis]{Analysis}

\section{Usability result and analysis}


\section{Performance result and analysis}


\begin{table}[]
\centering
\caption{Performance comparison HemeLB on Indy2 vs AWS EC2}
\label{table:perf}
\begin{tabular}{|c|c|c|}
\hline
\multicolumn{1}{|l|}{}            & \multicolumn{2}{c|}{Performance in seconds}               \\ \hline
\multicolumn{1}{|l|}{\# of Cores} & \multicolumn{1}{l|}{Indy2} & \multicolumn{1}{l|}{AWS EC2} \\ \hline
36                                & 24.7                       & 36.3                         \\ \hline
72                                & 12.7                       & 29.9                         \\ \hline
144                               & 7.08                       &                              \\ \hline
288                               & 3.44                       &                              \\ \hline
576                               & 1.81                       &                              \\ \hline
1152                              & 1.58                       &                              \\ \hline
\end{tabular}
\end{table}