% Activate the following line by filling in the right side. If for example the name of the root file is Main.tex, write
% "...root = Main.tex" if the chapter file is in the same directory, and "...root = ../Main.tex" if the chapter is in a subdirectory.
 
% !TEX root =  dissertation.tex

\chapter[Design and Specification]{Design and Specification}


I will discuss on the design and specification of HemeWeb. What are the resources needed to implement it and the design of the architecture.

\section{HemeWeb System Specification}


\vspace{1cm}

\noindent%
\begin{minipage}{\linewidth}% to keep image and caption on one page
\makebox[\linewidth]{
  \includegraphics[keepaspectratio=true,scale=0.5]{../resources/images/HemeWeb-phase-1.png}
 }
\captionof{figure}{Planned HemeWeb architecture phase 1 from \cite{Steven:2016aa}}\label{fig:hemeweb-phase-1}%      only if needed  
\end{minipage}

\vspace{1cm}

Figure \ref{fig:hemeweb-phase-1} above was the original plan on how the system architecture will look like. The system will consist mainly of one master instance which will take care of starting up HemeLB cluster to compute the simulation. This master instance will run HemeWeb web application in which the user will interact with. Users will provide the web application with the inputs it needs to run. Which are the geometry file (.gmy) and the job configuration file (.xml).. With this input, the master instance will start up the appropriate HemeLB core cluster and run mpirun command to initiate the job execution. When execution is done, the output will be available for download to the users. In addition to that, the job files will be uploaded to amazon S3 so that it will be available for other instances of HemeWeb if it is shared, or for newly created instance with the same amazon AWS account.


\section{HemeLB core container}

Part of the designed system above is to have HemeLB cluster available for the HemeWeb master instance. In order to spin up the cluster easily, I decided to make a HemeLB core container only available in the docker hub (https://hub.docker.com). This container is based on an earlier work to ship HemeLB and all of its setup tools in a docker container so peers can run HemeLB without configuring their system a lot. However, for HemeWeb purpose, the whole suit of the setup tools are not essential for the simulation. What I planned to do is to strip it down to its bare minimum, so the container only contains the binary necessary for the simulation to run.

